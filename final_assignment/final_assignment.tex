\documentclass[a4paper,12pt]{article}\usepackage[]{graphicx}\usepackage[]{color}
% maxwidth is the original width if it is less than linewidth
% otherwise use linewidth (to make sure the graphics do not exceed the margin)
\makeatletter
\def\maxwidth{ %
  \ifdim\Gin@nat@width>\linewidth
    \linewidth
  \else
    \Gin@nat@width
  \fi
}
\makeatother

\definecolor{fgcolor}{rgb}{0.345, 0.345, 0.345}
\newcommand{\hlnum}[1]{\textcolor[rgb]{0.686,0.059,0.569}{#1}}%
\newcommand{\hlstr}[1]{\textcolor[rgb]{0.192,0.494,0.8}{#1}}%
\newcommand{\hlcom}[1]{\textcolor[rgb]{0.678,0.584,0.686}{\textit{#1}}}%
\newcommand{\hlopt}[1]{\textcolor[rgb]{0,0,0}{#1}}%
\newcommand{\hlstd}[1]{\textcolor[rgb]{0.345,0.345,0.345}{#1}}%
\newcommand{\hlkwa}[1]{\textcolor[rgb]{0.161,0.373,0.58}{\textbf{#1}}}%
\newcommand{\hlkwb}[1]{\textcolor[rgb]{0.69,0.353,0.396}{#1}}%
\newcommand{\hlkwc}[1]{\textcolor[rgb]{0.333,0.667,0.333}{#1}}%
\newcommand{\hlkwd}[1]{\textcolor[rgb]{0.737,0.353,0.396}{\textbf{#1}}}%
\let\hlipl\hlkwb

\usepackage{framed}
\makeatletter
\newenvironment{kframe}{%
 \def\at@end@of@kframe{}%
 \ifinner\ifhmode%
  \def\at@end@of@kframe{\end{minipage}}%
  \begin{minipage}{\columnwidth}%
 \fi\fi%
 \def\FrameCommand##1{\hskip\@totalleftmargin \hskip-\fboxsep
 \colorbox{shadecolor}{##1}\hskip-\fboxsep
     % There is no \\@totalrightmargin, so:
     \hskip-\linewidth \hskip-\@totalleftmargin \hskip\columnwidth}%
 \MakeFramed {\advance\hsize-\width
   \@totalleftmargin\z@ \linewidth\hsize
   \@setminipage}}%
 {\par\unskip\endMakeFramed%
 \at@end@of@kframe}
\makeatother

\definecolor{shadecolor}{rgb}{.97, .97, .97}
\definecolor{messagecolor}{rgb}{0, 0, 0}
\definecolor{warningcolor}{rgb}{1, 0, 1}
\definecolor{errorcolor}{rgb}{1, 0, 0}
\newenvironment{knitrout}{}{} % an empty environment to be redefined in TeX

\usepackage{alltt}
\usepackage[a4paper,margin=2.54cm]{geometry}
\usepackage{titlesec}
\usepackage{enumitem}
\usepackage{parskip}
\usepackage{amssymb, amsmath, amsthm}
\usepackage{hyperref}
\usepackage{graphics}
\usepackage{graphicx}

\author{Tan Jun Yu (A0183140R)}

\title{%
	ST4234 Bayesian Statistics \\
	\large AY 19/20 Semester 2 Final Assignment}
\date{}

\newcommand{\Expect}{{\rm I\kern-.3em E}}
\newcommand{\by}{\mathbf{y}}
\newcommand{\Beta}{\mathrm{Beta}}
\newcommand{\N}{\mathrm{N}}
\newcommand{\Gammad}{\mathrm{Gamma}}
\newcommand{\Binomial}{\mathrm{Binomial}}
\newcommand{\Uniform}{\mathrm{Uniform}}
\newcommand{\id}{\mathrm{d}}
\IfFileExists{upquote.sty}{\usepackage{upquote}}{}
\begin{document}
\maketitle


\section*{Question 1}

\begin{enumerate}[label=(\alph*)]
\item We have $\by=\{y_1,\cdots,y_n\}$, $	p(y|\alpha,\beta)=\frac{\alpha\beta^\alpha}{(y+\beta)^{\alpha+1}},y>0,\alpha>0,\beta>0$.
$$p(\by|\alpha,\beta)=\prod_{i=1}^n\frac{\alpha\beta^\alpha}{(y_i+\beta)^{\alpha+1}}=(\alpha\beta^{\alpha})^n\prod_{i=1}^n\frac{1}{(y_i+\beta)^{\alpha+1}}$$
\item 
\begin{enumerate}[label=(\roman*)]
\item $\beta=1$.
\begin{align*}
p(\by|\alpha)&=\alpha^n\prod_{i=1}^n\frac{1}{(y_i+1)^{\alpha+1}}=\alpha^nc^{\alpha+1} \\
&\propto \alpha^n\exp\{\log(c^{\alpha+1})\} \\
&\propto \alpha^n\exp\{\alpha\log c\}
\end{align*}
Propose $\alpha\sim \Gammad(a,b)$ as the prior, so $p(\alpha)\propto \alpha^{a-1}\exp\{-b\alpha\}$.
\begin{align*}
p(\alpha|\by) &\propto p(\by|\alpha)p(\alpha) \\
&\propto \alpha^n\exp\{\alpha\log c\}\alpha^{a-1}\exp\{-b\alpha\} \\
&\propto \alpha^{n+a-1}\exp\{-\alpha(b-\log c)\}
\end{align*}
Therefore
$$\theta|\by\sim\Gammad(n+a,b-\log c), \textrm{where } c=\prod_{i=1}^n\frac{1}{(y_i+1)}$$
We can also do this via the one-parameter exponential family, any model whose density can be expressed as $p(y|\theta)=h(y)g(\theta)\exp\{\eta(\theta)t(y)\}$.
\begin{align*}
p(\by|\alpha)&=\alpha^n\prod_{i=1}^n\frac{1}{(y_i+1)^{\alpha+1}}=\alpha^nc^{\alpha+1} \\
&= \alpha^n\exp\left\{\log\left(\prod_{i=1}^n\frac{1}{(y_i+1)^{\alpha+1}}\right)\right\} \\
&= \alpha^n\exp\left\{-(\alpha+1)\sum_{i=1}^n\log(y_i+1)\right\}
\end{align*}
where $h(\by)=1,g(\alpha)=\alpha^n,\eta(\alpha)=-(\alpha+1),t(\by)=\sum_{i=1}^n\log(y_i+1)$. The conjugate prior is given by
\begin{align*}
p(\alpha)&\propto g(\alpha)^\nu\exp\{\eta(\alpha)\tau\} \\
&\propto (\alpha^n)^\nu\exp\{-(\alpha+1)\tau\} \\
&\propto \alpha^{n\nu}\exp\{-\alpha\tau\}\\
\alpha&\sim\Gammad(a,b),\textrm{\ where\ }a=n\nu+1,b=\tau 
\end{align*}
This conjugate prior is part of the $\Gammad(a,b)$ family, where $a=n\nu+1,b=\tau$.
The posterior density is
\begin{align*}
p(\alpha|\by)&\propto p(\by|\alpha)p(\alpha)\\
&=\alpha^n\exp\left\{-\alpha\sum_{i=1}^n\log(y_i+1)\right\}\alpha^{n\nu}\exp\{-\alpha\tau\}\\
&\propto \alpha^{n+n\nu}\exp\left\{-\alpha\left(\tau+\sum_{i=1}^n\log(y_i+1)\right)\right\}\\
\alpha|\by&\sim\Gammad\left(n+n\nu+1,\tau+\sum_{i=1}^n\log(y_i+1)\right)\\
&\sim \Gammad(n+a,\tau+t(\by))
\end{align*}
\item 
\begin{align*}
\ell(\alpha)&=\log p(\by|\alpha)=\log\left[\alpha^n\prod_{i=1}^n(y_i+1)^{-(\alpha+1)}\right] \\
&= n\log\alpha-(\alpha+1)\sum_{i=1}^n\log(y_i+1)\\
\ell'(\alpha) &= \frac{n}{\alpha}-\sum_{i=1}^n\log(y_i+1),\qquad \ell''(\alpha)=-\frac{n}{\alpha^2} \\
p(\alpha)&\propto\sqrt{I(\alpha)}=\sqrt{-\Expect_{y|\alpha}\left(-\frac{n}{\alpha^2}\right)}=\frac{n}{\alpha^2}\propto \frac{1}{\alpha}
\end{align*}
This is an improper prior as it does not integrate to 1.
\begin{align*}
p(\alpha|\by) &\propto p(\by|\alpha)p(\alpha) \\
&= \alpha^n\prod_{i=1}^n\frac{1}{(y_i+1)^{\alpha+1}}\times\frac{1}{\alpha} \\
&\propto \alpha^{n-1}c^\alpha
\end{align*}
Since $\int_{0}^{\infty}\alpha^ab^\alpha \id\alpha<\infty$, with $a=n-1\geq 0$, and $b=\prod_{i=1}^n\frac{1}{(y_i+1)^\alpha}$, the posterior is a proper probability distribution. Also, $p(\alpha|\by)\propto\alpha^{n-1}\exp\{\alpha\log c\}$, so $\alpha|\by\sim\Gammad(n,-\log c)$ which is a proper density.
\end{enumerate}
\item $\alpha=1$. We have 
$$p(\by|\beta)=\prod_{i=1}^n\frac{\beta}{(y_i+\beta)^2},\qquad p(\beta)=\frac{1}{(1+\beta)^2},\beta>0$$
$$p(\beta|\by)\propto p(\by|\beta)p(\beta)=\left(\prod_{i=1}^n\frac{\beta}{(y_i+\beta)^2}\right)\times \frac{1}{(1+\beta)^2}$$
Candidate proposal 1: $g(\beta)=\lambda\exp(-\lambda\beta)$

We want to find a finite positive bounding constant $M$ such that
$$\frac{p(\beta|\by)}{g(\beta)}=\frac{\left(\prod_{i=1}^n\frac{\beta}{(y_i+\beta)^2}\right)\frac{1}{(1+\beta)^2}}{\lambda\exp(-\lambda\beta)}\leq M$$
Both $p(\beta|\by)$ and $g(\beta)$ are decreasing functions, when $\beta\rightarrow\infty$, $g(\beta)$ decreases faster as it exponential, compared to $p(\beta|\by)$ which is polynomial. This means that for large values of $\beta$, we are not able to obtain a finite $M$, and the posterior density cannot be enveloped by the proposal density. Therefore this candidate is inappropriate.

Candidate proposal 2: $p(\beta)=\frac{1}{(1+\beta)^2}$

Again, we want to find a finite positive bounding constant $M$ such that
$$\frac{p(\beta|\by)}{p(\beta)}=\frac{\left(\prod_{i=1}^n\frac{\beta}{(y_i+\beta)^2}\right)\frac{1}{(1+\beta)^2}}{\frac{1}{(1+\beta)^2}}=\prod_{i=1}^n\frac{\beta}{(y_i+\beta)^2}\leq M$$
This likelihood function is positive. Since $y_i>0$ for all $i,\cdots,n$, $(y_i+\beta)^2>\beta$ for all $\beta>0$, this likelihood approaches $0^{+}$ when $\beta\rightarrow\infty$, indicating that the proposal has larger tails than the posterior. The proposal is appropriate and we will define $M$ as:
$$M=\frac{p(\beta|\by)}{p(\beta)}=\sup_{\beta}\prod_{i=1}^n\frac{\beta}{(y_i+\beta)^2}$$
We can obtain draws from $p(\beta|by)$ using the rejection sampling algorithm:
\begin{enumerate}[label=(\arabic*)]
	\item Generate $\beta^{(s)}\sim p(\beta)=\frac{1}{(1+\beta)^2}$
	\item Generate $U\sim\Uniform[0,1]$
	\item If $U<\frac{f(\beta^{(s)})}{Mg(\beta^{(s)})}$, accept $\beta^{(s)}$, otherwise reject $\beta^{(s)}$
	\item Repeat steps 1-3 until the desired sample $\left\{\beta^{(s)}:s=1,\cdots,S\right\}$ is obtained. The members of this sample will be random variables from $p(\beta|\by)$.
\end{enumerate}

\item Equation (2) gives us $p(y,\lambda|\alpha,\beta)=p(y|\lambda)p(\lambda|\alpha,\beta)$. To obtain Equation (1), which is $p(y|\alpha,\beta)$, we have to integrate $\lambda$ out of $p(y,\lambda|\alpha,\beta)$.
\begin{align*}
p(y|\alpha,\beta)&=\int p(y,\lambda|\alpha,\beta) \id\lambda \\
&= \int p(y|\lambda)p(\lambda|\alpha,\beta) \id\lambda \\
&= \int \lambda\exp(-\lambda y)\cdot \frac{\beta^\alpha}{\Gamma(\alpha)}\lambda^{\alpha-1}\exp(-\beta\lambda) \id\lambda\\
&= \frac{\beta^\alpha}{\Gamma(\alpha)}\int \lambda^\alpha\exp\{-\lambda(\beta+y)\} \id\lambda\\
&= \frac{\beta^\alpha}{\Gamma(\alpha)}\cdot\frac{\Gamma(\alpha+1)}{(\beta+y)^{\alpha+1}} \\
&= \frac{\alpha\beta^\alpha}{(y+\beta)^{\alpha+1}} \quad \textrm{\ which\ is\ Equation\ (1)}
\end{align*}
\end{enumerate}
\newpage
\section*{Question 2}
\begin{enumerate}[label=(\alph*)]
\item We have $p(y|\theta)=\theta(1-\theta)^y$ and $p(\theta)\propto \theta^{a-1}(1-\theta)^{b-1}$, so the posterior density is
\begin{align*}
p(\theta|\by) &\propto p(\by|\theta)p(\theta) \\
&\propto \left(\prod_{i=1}^n\theta(1-\theta)^{y_i}\right)\theta^{a-1}(1-\theta)^{b-1} \\
&\propto \theta^n(1-\theta)^{\sum_{i=1}^ny_i}\theta^{a-1}(1-\theta)^{b-1}
\end{align*}
\begin{align*}
\ell(\theta)=\log p(\theta|\by)&=(n+a-1)\log\theta+\left(b-1+\sum_{i=1}^ny_i\right)\log(1-\theta) \\
\ell'(\theta)&=\frac{n+a-1}{\theta}-\frac{b-1+\sum_{i=1}^ny_i}{1-\theta} \\
\ell'(\theta)&=0\implies \frac{n+a-1}{\theta}=\frac{b-1+\sum_{i=1}^ny_i}{1-\theta} \\
\hat\theta&=\frac{n+a-1}{n+a+b-2+\sum_{i=1}^ny_i} \\
\ell''(\theta)&=-\frac{n+a-1}{\theta^2}-\frac{b-1+\sum_{i=1}^ny_i}{(1-\theta)^2} \\
-\ell''(\theta)&=\frac{n+a-1}{\left(\frac{n+a-1}{n+a+b-2+\sum_{i=1}^ny_i}\right)^2}+\frac{b-1+\sum_{i=1}^ny_i}{\left(\frac{b-1+\sum_{i=1}^ny_i}{n+a+b-2+\sum_{i=1}^ny_i}\right)^2} \\
&= \frac{\left(n+a+b-2+\sum_{i=1}^ny_i\right)^2}{n+a-1}+\frac{\left(n+a+b-2+\sum_{i=1}^ny_i\right)^2}{b-1+\sum_{i=1}^ny_i} \\
&= \frac{\left(n+a+b-2+\sum_{i=1}^ny_i\right)^3}{(n+a-1)(b-1+\sum_{i=1}^ny_i)}
\end{align*}
The normal approximation to the posterior of $\theta$ is
$$\theta|\by\stackrel{\textrm{approx}}{\sim} \N\left(\frac{n+a-1}{n+a+b-2+\sum_{i=1}^ny_i},\frac{(n+a-1)(b-1+\sum_{i=1}^ny_i)}{\left(n+a+b-2+\sum_{i=1}^ny_i\right)^3}\right)$$
\item Take $g(\theta)=\theta$.
\begin{align*}
h^*(\theta)&=-n^{-1}[\log g(\theta)+\ell(\theta)] \\
&= -n^{-1}\left[\log\theta+(n+a-1)\log\theta+\left(b-1+\sum_{i=1}^ny_i\right)\log(1-\theta)\right] \\
&= -n^{-1}\left[(n+a)\log\theta+\left(b-1+\sum_{i=1}^ny_i\right)\log(1-\theta)\right]
\end{align*}
\begin{align*}
\frac{\id h^*(\theta)}{\id\theta} &= -n^{-1}\left[\frac{n+a}{\theta}-\frac{b-1+\sum_{i=1}^ny_i}{1-\theta}\right]\\
\frac{\id h^*(\theta)}{\id\theta}=0&\implies \frac{n+a}{\theta^*}=\frac{b-1+\sum_{i=1}^ny_i}{1-\theta} \\
\theta^*&=\frac{n+a}{a+b+n-1+\sum_{i=1}^ny_i} \\
\frac{\id^2 h^*(\theta)}{\id\theta^2} &= -n^{-1}\left[\frac{n+a}{\theta^2}+\frac{b-1+\sum_{i=1}^ny_i}{(1-\theta)^2}\right]
\end{align*}
\begin{align*}
\hat\sigma^{*2}&=\left[\frac{\id^2 h^*(\theta)}{\id\theta^2}\right]^{-1}\bigg\rvert_{\theta=\theta^*}=n\left[\frac{n+a}{\theta^{*2}}+\frac{b-1+\sum_{i=1}^ny_i}{(1-\theta^*)^2}\right]^{-1} \\
&=n\left[\frac{\left(a+b+n-1+\sum_{i=1}^ny_i\right)^2}{n+a}+\frac{\left(a+b+n-1+\sum_{i=1}^ny_i\right)^2}{b-1+\sum_{i=1}^ny_i}\right]^{-1} \\
&= n\frac{(n+a)(b-1+\sum_{i=1}^ny_i)}{\left(a+b+n-1+\sum_{i=1}^ny_i\right)^3}
\end{align*}
\begin{align*}
h(\theta)&=-n^{-1}\ell(\theta) \\
&= -n^{-1}\left[(n+a-1)\log\theta+\left(b-1+\sum_{i=1}^ny_i\right)\log(1-\theta)\right]
\end{align*}
\begin{align*}
\frac{\id h(\theta)}{\id\theta} &= -n^{-1}\left[\frac{n+a-1}{\theta}-\frac{b-1+\sum_{i=1}^ny_i}{1-\theta}\right]\\
\frac{\id h(\theta)}{\id\theta}=0&\implies \frac{n+a-1}{\hat\theta}=\frac{b-1+\sum_{i=1}^ny_i}{1-\hat\theta} \\
\hat\theta&=\frac{n+a-1}{a+b+n-2+\sum_{i=1}^ny_i} \\
\frac{\id^2 h(\theta)}{\id\theta^2} &= -n^{-1}\left[\frac{n+a-1}{\theta^2}+\frac{b-1+\sum_{i=1}^ny_i}{(1-\theta)^2}\right]
\end{align*}
\begin{align*}
\hat\sigma^{2}&=\left[\frac{\id^2 h(\theta)}{\id\theta^2}\right]^{-1}\bigg\rvert_{\theta=\hat\theta}=n\left[\frac{n+a-1}{\hat\theta^2}+\frac{b-1+\sum_{i=1}^ny_i}{(1-\hat\theta)^2}\right]^{-1} \\
&=n\left[\frac{\left(a+b+n-2+\sum_{i=1}^ny_i\right)^2}{n+a-1}+\frac{\left(a+b+n-2+\sum_{i=1}^ny_i\right)^2}{b-1+\sum_{i=1}^ny_i}\right]^{-1} \\
&= n\frac{(n+a-1)(b-1+\sum_{i=1}^ny_i)}{\left(a+b+n-2+\sum_{i=1}^ny_i\right)^3}
\end{align*}
\begin{align*}
\Expect(\theta|\by)&\approx\frac{\hat\sigma^*\theta^*p(\by|\theta^*)p(\theta^*)}{\hat\sigma p(\by|\hat\theta)p(\hat\theta)} \\
&=\frac{\sqrt{n\frac{(n+a)(b-1+\sum_{i=1}^ny_i)}{(a+b+n-1+\sum_{i=1}^ny_i)^3}}\cdot\theta^*\cdot(\theta^*)^{n+a-1}(1-\theta^*)^{b-1+\sum_{i=1}^ny_i}}{\sqrt{n\frac{(n+a-1)(b-1+\sum_{i=1}^ny_i)}{(a+b+n-2+\sum_{i=1}^ny_i)^3}}\cdot\hat\theta^{n+a-1}(1-\hat\theta)^{b-1+\sum_{i=1}^ny_i}} \\
&=\frac{(n+a)^{n+a+\frac{1}{2}}\left(a+b+n-2+\sum_{i=1}^ny_i\right)^{a+b+n-\frac{1}{2}+\sum_{i=1}^ny_i}}{(n+a-1)^{n+a-\frac{1}{2}}\left(a+b+n-1+\sum_{i=1}^ny_i\right)^{a+b+n+\frac{1}{2}+\sum_{i=1}^ny_i}}
\end{align*}
\item 
\begin{knitrout}\footnotesize
\definecolor{shadecolor}{rgb}{0.969, 0.969, 0.969}\color{fgcolor}\begin{kframe}
\begin{alltt}
\hlkwd{set.seed}\hlstd{(}\hlnum{0183140}\hlstd{)}

\hlstd{n} \hlkwb{<-} \hlnum{5}
\hlstd{sy} \hlkwb{<-} \hlnum{18}
\hlstd{a} \hlkwb{<-} \hlnum{2}
\hlstd{b} \hlkwb{<-} \hlnum{2}
\hlstd{S} \hlkwb{<-} \hlnum{100}

\hlstd{theta.samples} \hlkwb{<-} \hlkwd{runif}\hlstd{(S)}

\hlstd{logf} \hlkwb{<-} \hlkwa{function}\hlstd{(}\hlkwc{theta}\hlstd{,} \hlkwc{sy}\hlstd{,} \hlkwc{n}\hlstd{,} \hlkwc{a}\hlstd{,} \hlkwc{b}\hlstd{) \{}
        \hlstd{(n}\hlopt{+}\hlstd{a}\hlopt{-}\hlnum{1}\hlstd{)}\hlopt{*}\hlkwd{log}\hlstd{(theta)} \hlopt{+} \hlstd{(b}\hlopt{-}\hlnum{1}\hlopt{+}\hlstd{sy)}\hlopt{*}\hlkwd{log}\hlstd{(}\hlnum{1}\hlopt{-}\hlstd{theta)}
\hlstd{\}}

\hlstd{logw} \hlkwb{<-} \hlkwd{logf}\hlstd{(theta.samples, sy, n, a, b)} \hlopt{-}
        \hlkwd{dunif}\hlstd{(theta.samples,} \hlkwc{log} \hlstd{=} \hlnum{TRUE}\hlstd{)}
\hlstd{w} \hlkwb{<-} \hlkwd{exp}\hlstd{(logw} \hlopt{-} \hlkwd{max}\hlstd{(logw))}
\hlstd{W} \hlkwb{<-} \hlstd{w}\hlopt{/}\hlkwd{sum}\hlstd{(w)}
\hlstd{(est} \hlkwb{<-} \hlkwd{sum}\hlstd{(W}\hlopt{*}\hlstd{theta.samples))}
\end{alltt}
\begin{verbatim}
## [1] 0.2675
\end{verbatim}
\begin{alltt}
\hlstd{(se} \hlkwb{<-} \hlkwd{sqrt}\hlstd{(}\hlkwd{sum}\hlstd{((W}\hlopt{*}\hlstd{(theta.samples} \hlopt{-} \hlstd{est))}\hlopt{^}\hlnum{2}\hlstd{)))}
\end{alltt}
\begin{verbatim}
## [1] 0.0099
\end{verbatim}
\end{kframe}
\end{knitrout}
From the output above, the important sampling estimate of $\Expect(\theta|by)=0.2675$ and with standard error of 0.0099.
\end{enumerate}
\newpage
\section*{Question 3}
\begin{enumerate}[label=(\alph*)]
\item The unnormalized posterior density of $(\theta_1,\theta_2)$ is 
\begin{align*}
p(\theta_1,\theta_2|\by)&\propto p(\by|\theta_1,\theta_2)p(\theta_1,\theta_2) \\
&\propto \left(\frac{\theta_1}{4}+\frac{1}{8}\right)^{y_1}\left(\frac{\theta_1}{4}\right)^{y_2}\left(\frac{\theta_2}{4}\right)^{y_3}\left(\frac{\theta_2}{4}+\frac{3}{8}\right)^{y_4}\left(\frac{1-\theta_1-\theta_2}{2}\right)^{y_5} \\
&\propto \left(\frac{2\theta_1+1}{8}\right)^{y_1}\left(\frac{\theta_1}{4}\right)^{y_2}\left(\frac{\theta_2}{4}\right)^{y_3}\left(\frac{2\theta_2+3}{8}\right)^{y_4}\left(\frac{1-\theta_1-\theta_2}{2}\right)^{y_5} \\
&\propto (2\theta_1+1)^{y_1}\theta_1^{y_2}\theta_2^{y_3}(2\theta_2+3)^{y_4}(1-\theta_1-\theta_2)^{y_5}
\end{align*}
\item We split the first and fourth category into two new categories each, augmenting the data into seven categories in total, with probabilities
$$\left(\frac{\theta_1}{4},\frac{1}{8},\frac{\theta_1}{4},\frac{\theta_2}{4},\frac{\theta_2}{4},\frac{3}{8},\frac{1-\theta_1-\theta_2}{2}\right)$$
Let the counts of the first two categories be $h,y_1-h$ and the counts of the fifth and sixth categories be $k,y_4-k$, so we have
$$\by=(h,y_1-h,y_2,y_3,k,y_4-k,y_5)$$
The full posterior is then
\begin{align*}
p(\theta_1,\theta_2,h,k|\by)&=\frac{p(\theta_1,\theta_2,h,k,\by)}{p(\by)}\propto p(\theta_1,\theta_2,h,k,\by)\\
&=p(y,h,k|\theta_1,\theta_2)p(\theta_1,\theta_2) \\
&= p(h,y_1-h,y_2,y_3,k,y_4-k,y_5|\theta_1,\theta_2)\times 1 \\
&=\frac{n!}{h!(y_1-h)!y_2!y_3!k!(y_4-k)!y_5!}\times \\
&\quad\left(\frac{\theta_1}{4}\right)^{h}\left(\frac{1}{8}\right)^{y_1-h}\left(\frac{\theta_1}{4}\right)^{y_2}\left(\frac{\theta_2}{4}\right)^{y_3}\left(\frac{\theta_2}{4}\right)^{k}\left(\frac{3}{8}\right)^{y_4-k}\left(\frac{1-\theta_1-\theta_2}{2}\right)^{y_5} \\
&\propto \frac{1}{h!(y_1-h)!k!(y_4-k)!}\times\\
&\quad \left(\frac{1}{8}\right)^{y_1-h}\left(\frac{3}{8}\right)^{y_4-k}\left(\frac{\theta_1}{4}\right)^{h+y_2}\left(\frac{\theta_2}{4}\right)^{y_3+k}\left(\frac{1-\theta_1-\theta_2}{2}\right)^{y_5}
\end{align*}
We find the four conditional posteriors:
\begin{align*}
p(\theta_1|h,k,\theta_2,y)&\propto \left(\frac{\theta_1}{4}\right)^{h+y_2}\left(\frac{1-\theta_1-\theta_2}{2}\right)^{y_5}\\
&\propto \theta_1^{h+y_2}(1-\theta_2-\theta_1)^{y_5} \\
\implies \theta_1|h,k,\theta_2,\by&\sim (1-\theta_2)S_1,\quad S_1\sim\Beta(h+y_2+1,y_5+1)
\end{align*}
\begin{align*}
p(\theta_2|h,k,\theta_1,y)&\propto \left(\frac{\theta_2}{4}\right)^{y_3+k}\left(\frac{1-\theta_1-\theta_2}{2}\right)^{y_5}\\
&\propto \theta-2^{y_3+k}(1-\theta_2-\theta_1)^{y_5}\\
\implies \theta_2|h,k,\theta_1,\by&\sim (1-\theta_1)S_2,\quad S_2\sim\Beta(y_3+k+1,y_5+1)
\end{align*}
\begin{align*}
p(h|k,\theta_1,\theta_2,\by)&\propto \frac{1}{h!(y_1-h)!}\left(\frac{1}{8}\right)^{y_1-h}\left(\frac{\theta_1}{4}\right)^{h+y_2}\\
&\propto \binom{y_1}{h}\left(\frac{1}{8}\right)^{y_1-h}\left(\frac{\theta_1}{4}\right)^{h}\\
&\propto \binom{y_1}{h}\left(\frac{\frac{1}{8}}{\frac{1}{8}+\frac{\theta_1}{4}}\right)^{y_1-h}\left(\frac{\frac{\theta_1}{4}}{\frac{1}{8}+\frac{\theta_1}{4}}\right)^{h}\\
&\propto \binom{y_1}{h}\left(\frac{1}{1+2\theta_1}\right)^{y_1-h}\left(\frac{2\theta_1}{1+2\theta_1}\right)^{h}\\
\implies h|k,\theta_1,\theta_2,\by&\sim \Binomial\left(y_1,\frac{2\theta_1}{1+2\theta_1}\right)
\end{align*}
\begin{align*}
p(k|h,\theta_1,\theta_2,\by)&\propto \frac{1}{k!(y_4-k)!}\left(\frac{3}{8}\right)^{y_4-k}\left(\frac{\theta_2}{4}\right)^{y_3+k}\\
&\propto \binom{y_4}{k}\left(\frac{3}{8}\right)^{y_4-k}\left(\frac{\theta_2}{4}\right)^{k}\\
&\propto \binom{y_4}{k}\left(\frac{\frac{3}{8}}{\frac{3}{8}+\frac{\theta_2}{4}}\right)^{y_4-k}\left(\frac{\frac{\theta_2}{4}}{\frac{3}{8}+\frac{\theta_2}{4}}\right)^{k}\\
&\propto \binom{y_4}{k}\left(\frac{3}{3+2\theta_2}\right)^{y_4-k}\left(\frac{2\theta_2}{3+2\theta_2}\right)^k\\
\implies k|h,\theta_1,\theta_2,\by&\sim\Binomial\left(y_4,\frac{2\theta_2}{3+2\theta_2}\right)
\end{align*}
Now we can use Gibbs sampler to draw $T$ samples from the posterior $p(\theta_1,\theta_2|y)$.
\begin{enumerate}[label=(\roman*)]
	\item Initialize $(\theta_1^{(0)},\theta_2^{(0)})$ such that $0\leq\theta_1^{(0)}\leq1,\quad0\leq\theta_2^{(0)}\leq1,\quad0\leq\theta_1^{(0)}+\theta_2^{(0)}\leq1$
	\item For $t=1,\cdots,T$,
	\begin{enumerate}[label=(\arabic*)]	
		\item Draw $h^{(t)}$ from $\Binomial\left(y_1,\frac{2\theta_1^{(t-1)}}{1+2\theta_1^{(t-1)}}\right)$
		\item Draw $k^{(t)}$ from $\Binomial\left(y_4,\frac{2\theta_2^{(t-1)}}{3+2\theta_2^{(t-1)}}\right)$
		\item Draw $\theta_1^{(t)}$ from $(1-\theta_2^{(t-1)})\Beta(h^{(t)}+y_2+1,y_5+1)$
		\item Draw $\theta_2^{(t)}$ from $(1-\theta_1^{(t-1)})\Beta(y_3+k^{(t)}+1,y_5+1)$
	\end{enumerate}
\end{enumerate}
\end{enumerate}
\newpage
\section*{Question 4}
\begin{enumerate}[label=(\alph*)]
\item We have $y_i\stackrel{\mathrm{iid}}{\sim} \Binomial(n_i,p_i)$ and $p(\beta_0,\beta_1)\propto 1$. Also,
$$\log\frac{p_i}{1-p_i}=\beta_0+\beta_1x_i \implies
p_i = \frac{\exp(\beta_0+\beta_1x_i)}{1+\exp(\beta_0+\beta_1x_i)}$$
The posterior density is
\begin{align*}
p(\beta_0,\beta_1|\by)&\propto p(\by|\beta_0,\beta_1)p(\beta_0,\beta_1) \\
&\propto \prod_{i=1}^n\left(\frac{\exp(\beta_0+\beta_1x_i)}{1+\exp(\beta_0+\beta_1x_i)}\right)^{y_i}\left(\frac{1}{1+\exp(\beta_0+\beta_1x_i)}\right)^{n_i-y_i}
\end{align*}
The log posterior density is
\begin{align*}
\log p(\beta_0,\beta_1|y) &\propto \sum_{i=1}^ny_i\left[(\beta_0+\beta_1x_i)-\log(1+\exp(\beta_0+\beta_1x_i)\right]\\
&\qquad\quad+(n_i-y_i)\left[-\log(1+\exp(\beta_0+\beta_1x_i))\right] \\
&\propto \sum_{i=1}^n\left\{y_i(\beta_0+\beta_1x_i)-n_i\log(1+\exp(\beta_0+\beta_1x_i))\right\}
\end{align*}
\item The covariance for the normal approximation is the negative inverse of the Hessian from the output.
\begin{align*}
\det(H)&=-17.4066\times-9017.5273 -(-393.5803\times -393.5803) \\
&= 2059.038152
\end{align*}
\begin{align*}
-\begin{bmatrix}
-17.4066 & -393.5803 \\
-393.5803 & -9017.5273
\end{bmatrix}^{-1}&=\frac{1}{2059.038152}
\begin{bmatrix}
9017.5273 & -393.5803 \\
-393.5803 & 17.4066
\end{bmatrix} \\
&= \begin{bmatrix}
4.3794963 & -0.191148133 \\
-0.1911481 & 0.008453774
\end{bmatrix}
\end{align*}
The normal approximation to the posterior $p(\beta_0,\beta_1|\by)$ is
$$\beta_0,\beta_1|\by\sim\N\left(\begin{bmatrix}
-5.366289 \\
0.238423
\end{bmatrix},\begin{bmatrix}
4.3794963 & -0.191148133 \\
-0.1911481 & 0.008453774
\end{bmatrix}\right)$$
\item 
\begin{enumerate}[label=(\roman*)]
	\item Algorithm 2 uses a scale of 4 but Algorithm 1 has no scaling factor. This allows the random walk in Algorithm 1 to take larger steps, converging towards the posterior mode using fewer iterations.
	\item Algorithm 2 uses the covariance matrix in the normal approximation, so it models the posterior more closely and hence its acceptance rate will be closer to the optimal acceptance rate. We learnt that the optimal acceptance rate of a random walk Metropolis can be achieved by setting $g(\beta^*|\beta^{(t-1)})$ as the density $\N(\theta^{(t-1)},c^2(-H)^{-1})$, where $H$ is the Hessian and $c=2.4/\sqrt{d}$, for dimensionality of $\beta$,$d$, where $d\geq 3$.
\end{enumerate}
Algorithm 1 assumes independence in $(\beta_2,\beta_1)$, but as we can observe from the covariance matrix in the normal approximation, this is not actually the case. Each sample for $\beta_0$ will affect $\beta_1$ and vice versa. If we use Algorithm 1, we will converge slowly to the posterior mode, because at each iteration $\beta_0$ and $\beta_1$ would not move in an optimal direction towards the posterior mode. This is also exacerbated by the fact that Algorithm 1 takes smaller steps for each iteration. Another downside of this is that Algorithm 1 will not explore the parameter space as much as Algorithm 2, which limits its ability to identify potential neighbouring peaks.

\item We can observe drastic changes in values for $\beta_0$ and $\beta_1$ for the first 200 or so iterations, these values are not distributed from the posterior distribution, but initial steps to approach the posterior mode given the starting values. We can discard the first 500 values as burn in. Alternatively, since starting values of around $(0,-1.25)$ were used according to the plots, we can use choose more appropriate starting values: the estimated posterior mode itself using \verb|out$par| from the given output.

The traceplots do not show wild or rapid fluctuation; there are observable upward and downward trends. Thus, there are considerable serial dependence in both $\beta_0$ and $\beta_1$, indicating that the Markov chain is not mixing well. We can use a larger scale so that wider steps are taken each iteration to induce greater fluctuation. Alternatively, we can also use the method of ``thinning'' only keep the sample at every $n$th iteration to reduce autocorrelation. However, this may lead to inefficiency without solving the root problem of insufficient step size.
\end{enumerate}
\end{document}
